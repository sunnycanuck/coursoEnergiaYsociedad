\documentclass[11pt]{article}
\usepackage[spanish,mexico]{babel}
\usepackage{multicol}
\usepackage{hyperref}
\usepackage{amsmath}

\begin{document}

\title{Lectura 3}
\author{Rodger Evans}
\date{ May 7, 2013 }
\maketitle

\section{Termodynamica}

\subsection{Unidades b\'asicas del Sistema Internacional}

Hay siete unidades b\'asicas de donde todos los otras unidades solamente non derivaba. Cada unidad tienes su propio dimension. 
\\
\\
\begin{tabular}{c c c}
Cantidad b\'asica & simbolo para cantidad & simbolo para dimension \\
\hline
longitud & $l$ & $ \boldsymbol{L}$\\
masa & $m$ & $ \boldsymbol{M}$\\
tiempo & $t$ & $\boldsymbol{T}$\\
coriente electrica & $I$ & $\boldsymbol{I}$\\
temperatura termodinamica & $T$& $ \boldsymbol{T}$\\
cantidad de substancia & $n$ & $ \boldsymbol{N}$\\
intensidad luminosa & $I_v$ & $ \boldsymbol{J}$\\
\hline
\end{tabular}


\subsection{Propiedades intensivos y extensivos}

\subsubsection{Intesivos}

Propiedades intr\'insecas son inherente de la sistema como densidad; es independente de la cantidad de la material y su forma y son mas dependente en su compsistion quemica o estructrual. 
Extrinsica son propiedades que viene de aspectos exterior de la sistema como peso y volume.

El relaci\'on de dos propiedades extrinisca da una propiedad intr\'inseca como el masa sobre el volumen que es el densidad y es invariante en escala, entonces intr\'inseca. 

\begin{multicols}{2}
\begin{itemize}
    \item potencial quimecia
    \item capacidad calor\'ifica
    \item \href{https://en.wikipedia.org/wiki/Malleability}{ductilidad}
    \item desidad o gravidade espcifica
    \item \href{https://en.wikipedia.org/wiki/Hardness}{dureza}
    \item \href{https://en.wikipedia.org/wiki/Magnetization}{magnetizaci\'on}
    \item viscosidad
    \item \href{https://en.wikipedia.org/wiki/Melting_point}{Punto de fusi\'on}
    \item \href{https://en.wikipedia.org/wiki/Concentration}{Concentraci\'on}
    \item restividad electrica
    \item \href{https://en.wikipedia.org/wiki/Pressure}{Presi\'on}
    \item \href{https://en.wikipedia.org/wiki/Temperature}{temperatura}
    \item espectral máximos de absorción
    \item \href{https://en.wikipedia.org/wiki/Specific_volume}{volumen específico}
    \item energia especifica
    \item elasticidad
\end{itemize}
\end{multicols}

\subsubsection{Extensivos}

Este es una cantidad que es aditivo para subsistemas independientes que no interactúan. 
Es para decir, si tienes una manzana y una naraga, el masa de los dos es una proiedad extensivo. 

Unas ejemplos de propiedades extensivos son:

\begin{multicols}{2}
\begin{itemize}
	\item energ\'ia
	\item masa
	\item volume
	\item entropia
	\item numero de particulos
	\item carga electrica
	\item energ\'ia Gibbs
	\item momento 
	\item peso
	\item longitud
\end{itemize}
\end{multicols}

Masa es el cantidad de material en una objecto. Peso es la fuerza que siente la masa en una campo gravitacional.

\begin{equation}
	F_g=m g
\end{equation}

Donde $F_g$ es el fuerza de gravedad, $m$ es la masa y $g$ es el constante local de gravedad. 

\subsection{Volumen específico}

El volumen especifico es el relación a la volumen y masa de la sistema. Es el inversa de dessidad, y es una propiedad intr\'inseca de material.

%\\
%\cite{Lemenand10}

%\bibliographystyle{plain}
%\bibliography{SustainableBibTeX}

\end{document}